\cs{Заключение}

В рамках квалификационной работы была поставлена цель выбрать оптимальную модификацию алгоритма глобального поиска (AGS)
и на его основе построить метод для одновременного решения множества задач глобальной оптимизации с нелинейными ограничениями.

В ходе работы получены следующие результаты:
\begin{enumerate}
    \item Произведено сравнение различных способов редукции размерности, основанных на отображениях типа кривой Пеано.
    В результате был сделан вывод о том, что для построения базовой многомерной версии метода AGS 
    с практической стороны наиболее выгодно использовать единственную кривую Пеано;
    \item Предложена модификация AGS, AGS-AR, менее чувствительная к параметрам, и сходящаяся так же быстро, как AGS
    с заранее подобранными под класс задач параметрами;
    \item AGS-AR продемонстрировал надёжность и скорость сходимости на уровне другого детерминированного метода, DIRECT,
    и превзошёл на рассмотренных тестовых задачах многие другие методы, реализации которых так же доступны в исходных кодах;
    \item Программная реализация метода AGS-AR прошла процедуру ревью и была включена в состав популярной библиотеки
    алгоритмов нелинейной оптимизации NLOpt;
    \item Реализована поддержка нелинейных ограничений в алгоритме, решающeм
    множество задач глобальной оптимизации в совокупности и распределяющего свои ресурсы так, чтобы
    обеспечивать равномерную сходимость во всех задачах. Доказано теорема о достаточных условиях сходимости
    полученного метода. Свойство равномерной сходимости проверено с помощью численного эксперимента.
    \item В ходе численных экспериментов была оценена эффективность выполненной параллельной реализации метода, решающего множество задач, и
    показана эффективность такого подхода над решением задач по отдельности на примере получения Парето фронта в многокритериальной задаче с
    нелинейными ограничениями.
\end{enumerate}
