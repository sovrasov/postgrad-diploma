\cs{Введение}

\subsection*{Актуальность темы исследования}

Задачи нелинейной глобальной оптимизации встречаются в различных прикладных областях \cite{Kvasov2013, Barkalov2013},
и традиционно считаются одними из самых трудоёмких среди оптимизационных задач.
Их сложность экспоненциально растёт в зависимости от размерности пространства поиска \cite{Vavasis1995}.
Зачастую глобальная оптимизация при размерности пространства в 10 переменных сложнее,
чем локальная оптимизация в существенно многомерном пространстве.
Для последней может оказаться достаточно применения простейшего метода градиентного спуска
или эвристических алгоритмов поиска по шаблону \cite{torczon1997},
в то время как чтобы \textit{гарантированно} отыскать глобальный оптимум, методам
оптимизации приходится накапливать информацию о поведении целевой функции во всей области поиска
\cite{Jones2009,Paulavicius2011,Evtushenko2013,Strongin2000}.

На практике в роли параметров для оптимизации могут выступать параметры математической модели, например, при решении
задачи поиска оптимального управления или при настройке алгоритма машинного обучения под конкретные данные.
В этих случаях о поведении целевой функции (выпуклость, наличие производных) ничего неизвестно, т.е. метод оптимизации
работает с функцией, представляющей собой чёрный ящик. Кроме того, подобный чёрный ящик, как правило,
имеет высокую вычислительную сложность, во много раз превосходящую вычислительную сложность решающих правил метода оптимизации.

Широко представленным классом методов, решающих задачи с трудоёмкой целевой функцией типа чёрный ящик, являются
алгоритмы липшицевой глобальной оптимизации. Они требуют, чтобы целевая функция удовлетворяла условию Липшица в области поиска,
что является более сильным свойством, чем непрерывность, но более слабым, чем предположение о наличии непрерывных частных производных.
В то же время, этого достаточно, чтобы гарантировать глобальную сходимость при условии выбора
подходящих параметров применяемого метода оптимизации.

В задачах глобальной оптимизации область допустимых значений целевой функции также может быть определена
с помощью условий, например в виде функциональных ограничений типа неравенство. Это вносит дополнительную сложность в процесс решения
задач, требуя создания схем учёта ограничений. Ещё более сложными являются задачи многокритериальной условной многокритериальной
оптимизации и смешанного программирования. В последнем случае как целевая функция, так и функциональные ограничения
зависят и от непрерывных, и от дискретных переменных. Существуют методы скаляризации многокритериальных задач.
При этом порождается множество скалярных задач, каждую из которых необходимо решить. Задачи смешанного программирования
в некоторых случаях тоже можно свести к серии задач, каждая из которых соответствует одному дискретному параметру.
Решение серии задач условной глобальной задач при ограниченных вычислительных
ресурсах налагает дополнительные требования на метод оптимизации: помимо поиска глобального экстремума необходимо
распределять вычислительные ресурсы так, чтобы сразу во всех решаемых задачах положение глобального
экстремума было оценено примерно с одинаковым качеством.

Многообразие постановок задач и методов их решения говорит о том, что глобальная оптимизация
является развивающимся и актуальным направлением исследований в теории оптимизации, широко применяемым на практике.

%Обычно серию из \(q\) задач решают либо последовательно, либо
%параллельно порциями по \(p,\:p<<q\) задач, где \(p\) --- количество параллельных вычислительных устройств.
%Такой подход ведет к тому, что в каждый момент времени до окончания вычислений
%остаются задачи, в которых оценка глобального оптимума не получена вообще, в то время, как в задачах из начала
%списка оптимум может быть оценен даже с избыточной точностью.

%В данной практической работе рассматривается обобщение ранее разработанного в ННГУ им. Н. И. Лобачевского
% параллельного метода глобальной оптимизации для одновременного решения множества задач \cite{BarkalovStrongin2018} на
% случай задач с нелинейными ограничениями. Для учета ограничений используется индексная схема \cite{Strongin2000},
% позволяющая работать с частично вычислимыми целевым функциями и обладающая экономичностью,
% сравнимой с другими подходами \cite{BarkalovLebedev2017}. Эффективность реализованного
% алгоритма показана на примере решения множеств задач, сгенерированных специализированным
% механизмом, порождающим наборы задач заданной размерности с заданным количеством нелинейных ограничений \cite{GergelBarkalov2019}.
% Кроме искусственно сгенерированных задач, рассматриваемый метод протестирован также
% на множестве задач, возникающем при решении задачи многокритериальной оптимизации
% с нелинейными ограничениями методом свертки критериев \cite{Ehrgott2005}.


\subsection*{Цель квалификационной работы}

Целью квалификационной работы является исследование и разработка последовательных и параллельных алгоритмов для
решения задач глобальной оптимизации, в частности разработка метода, который бы эффективно распределял ресурсы между задачами при
решении множества задач условной оптимизации.

Для достижения поставленной цели, необходимо решить следующие задачи:
\begin{enumerate}
    \item Сделать обзор современных методов глобальной оптимизации;
    \item Сравнить актуальные методы между собой;
    \item Сформулировать собственную модификацию алгоритма, решающего множество задач условной глобальной оптимизации,
    получить условия сходимости;
    \item Оценить качество работы полученного метода на большом наборе разнообразных тестовых задач с ограничениями.
\end{enumerate}

\textbf{Объект исследования} -- программные системы для решения задач глобальной оптимизации в различных постановках.

\textbf{Предмет исследования} -- методы решения задач условной глобальной оптимизации, теория их сходимости.

\subsection*{Обоснование специальности}

Область исследования соответствует следующим пунктам паспорта специальности 05.13.18 --
«Математическое моделирование, численные методы и комплексы программ»:
4. Реализация эффективных численных методов и алгоритмов в виде
комплексов проблемно-ориентированных программ для проведения
вычислительного эксперимента; 3. Разработка, обоснование и тестирование эффективных вычислительных
методов с применением современных компьютерных технологий.

\subsection*{Научная новизна}

В процессе работы была, разработана модификация метода для решения множества задач условной оптимизации, поддерживающая
функциональные ограничения типа неравенство.
Новизна работы заключается в следующем:

\begin{enumerate}
    \item Предложена модификация алгоритма AGS, которая менее чувствительна к параметрам и сходится так же быстро, как AGS
    с заранее подобранными под класс задач параметрами;
    \item Реализована поддержка нелинейных ограничений в алгоритме, решающeм
    множество задач глобальной оптимизации в совокупности и распределяющего свои ресурсы так, чтобы
    обеспечивать равномерную сходимость во всех задачах. Доказано теорема о достаточных условиях сходимости
    полученного метода. Свойство равномерной сходимости проверено с помощью численного эксперимента.
\end{enumerate}

\subsection*{Обоснованность и достоверность результатов}
Достоверность результатов, полученных в ходе выполнения квалификационной работы, подтверждается
тем, что различные части работы были опубликованы в рецензируемых сборниках трудов российских и международных конференций.
Кроме того, программная реализация метода AGS-AR прошла процедуру ревью и была включена в состав популярной библиотеки
алгоритмов нелинейной оптимизации NLOpt.

\textbf{Публикации.} Основные результаты по теме квалификационной работы изложены в трёх публикациях \cite{barkalovSovraov2019,sovrasov2019,sovrasov2020},
каждая из которых входит в сборник трудов конференций, индексируемый в базе данных Scopus.

\subsection*{Структура и объемы работы}

Квалификационная работа включает в себя введение, три главы, заключение и список литературы.
Полный объём квалификационной работы составляет 38 страниц, включая 11 рисунков и 6 таблиц.
Список литературы содержит 49 позиций.

\subsection*{Личный вклад автора}

Личный вклад автора состоит в следующем:

\begin{enumerate}
    \item Разработка методов, теоретический анализ их сходимости, их программная реализация.
    \item Разработка плана вычислительных экспериментов, их проведение, анализ результатов.
    \item Подготовка публикаций по проделанной работе.
\end{enumerate}