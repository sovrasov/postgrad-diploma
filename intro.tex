\section*{Введение}
\addcontentsline{toc}{section}{Введение}


Задачи нелинейной глобальной оптимизации встречаются в различных прикладных областях \cite{Kvasov2013, Barkalov2013},
и традиционно считаются одними из самых трудоёмких среди оптимизационных задач.
Их сложность экспоненциально растёт в зависимости от размерности пространства поиска \cite{Vavasis1995}.
Зачастую глобальныя оптимизация при размерности пространства в 10 переменных сложнее,
чем локальная оптимизация в существенно многомерном пространстве.
Для последней может оказаться достаточно применения простейшего метода градиентного спуска
или эвристических алгоритмов поиска по шаблону \cite{torczon1997},
в то время как чтобы \textit{гаранитрованно} отыскать глобальный оптимум методам
оптимизации приходится накапливать информацию о поведении целевой функции во всей области поиска
\cite{Jones2009,Paulavicius2011,Evtushenko2013,Strongin2000}.


% В последнее время стали популярны различные стохастические алгоритмы глобально оптимизации,
% прежде всего эволюционные \cite{Storn1997, SCHLUTER2009, KennedyEberhart1995}. 
% Они имеют довольно простую структуру, позволяют решать задачи большой размерности,
% однако обеспечивают глобальную сходимость только в вероятностном смысле.

% В данной работе рассмотрены open-source реализации восьми различных методов глобальной оптимизации, представленные
% в библиотеке NLOpt\cite{nlopt} и пакете SciPY\cite{scipy}.
% Все алгоритмы были протестированы на наборе из 900 существенно многоэкстремальных функций, который был сгенерирован с
% помощью специализированных генераторов задач \cite{Gaviano2003, grishaginClass}.


\addcontentsline{toc}{subsection}{Актуальность темы исследования}
\subsection*{Актуальность темы исследования}

\addcontentsline{toc}{subsection}{Цель диссертационной работы}
\subsection*{Цель диссертационной работы}


Объект исследования –

Предмет исследования – м

Обоснование специальности

Научная новизна

Практическая ценность работы

Обоснованность и достоверность результатов

Основные положения, выносимые на защиту

Реализация результатов работы

Структура и объемы работы

Личный вклад автора
