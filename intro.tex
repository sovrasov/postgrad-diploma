\cs{Введение}

Задачи нелинейной глобальной оптимизации встречаются в различных прикладных областях \cite{Kvasov2013, Barkalov2013},
и традиционно считаются одними из самых трудоёмких среди оптимизационных задач.
Их сложность экспоненциально растёт в зависимости от размерности пространства поиска \cite{Vavasis1995}.
Зачастую глобальная оптимизация при размерности пространства в 10 переменных сложнее,
чем локальная оптимизация в существенно многомерном пространстве.
Для последней может оказаться достаточно применения простейшего метода градиентного спуска
или эвристических алгоритмов поиска по шаблону \cite{torczon1997},
в то время как чтобы \textit{гарантированно} отыскать глобальный оптимум, методам
оптимизации приходится накапливать информацию о поведении целевой функции во всей области поиска
\cite{Jones2009,Paulavicius2011,Evtushenko2013,Strongin2000}.

Широким и хорошо изученным классом

Кроме сложности построения методов глобальной оптимизации, гарантирующих сходимость, существует
проблема

Решение серии таких задач при ограниченных вычислительных
ресурсах является еще более сложной задачей: помимо поиска глобального экстремума необходимо
распределять вычислительные ресурсы так, чтобы сразу во всех решаемых задачах положение глобального
экстремума было оценено примерно с одинаковым качеством. Обычно серию из \(q\) задач решают либо последовательно, либо
параллельно порциями по \(p,\:p<<q\) задач, где \(p\) --- количество параллельных вычислительных устройств.
Такой подход ведет к тому, что в каждый момент времени до окончания вычислений
остаются задачи, в которых оценка глобального оптимума не получена вообще, в то время, как в задачах из начала
списка оптимум может быть оценен даже с избыточной точностью.

В данной практической работе рассматривается обобщение ранее разработанного в ННГУ им. Н. И. Лобачевского
параллельного метода глобальной оптимизации для одновременного решения множества задач \cite{BarkalovStrongin2018} на
случай задач с нелинейными ограничениями. Для учета ограничений используется индексная схема \cite{Strongin2000},
позволяющая работать с частично вычислимыми целевым функциями и обладающая экономичностью,
сравнимой с другими подходами \cite{BarkalovLebedev2017}. Эффективность реализованного
алгоритма показана на примере решения множеств задач, сгенерированных специализированным
механизмом, порождающим наборы задач заданной размерности с заданным количеством нелинейных ограничений \cite{GergelBarkalov2019}.
Кроме искусственно сгенерированных задач, рассматриваемый метод протестирован также
на множестве задач, возникающем при решении задачи многокритериальной оптимизации
с нелинейными ограничениями методом свертки критериев \cite{Ehrgott2005}.



\subsection*{Актуальность темы исследования}

\subsection*{Цель диссертационной работы}


Объект исследования –

Предмет исследования – м

\subsection*{Обоснование специальности}

\subsection*{Научная новизна}

\subsection*{Практическая ценность работы}

\subsection*{Обоснованность и достоверность результатов}

% Основные положения, выносимые на защиту

\subsection*{Реализация результатов работы}

\subsection*{Структура и объемы работы}

Квалификационная работа включает в себя введение, три главы, заключение и список литературы.
Полный объём квалификационной работы составляет 35 страниц, включая 11 рисунков и 6 таблиц.
Список литературы содержит 37 позиций.

\subsection*{Личный вклад автора}
