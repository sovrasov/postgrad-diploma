\cs{Введение}

Задачи нелинейной глобальной оптимизации встречаются в различных прикладных областях \cite{Kvasov2013, Barkalov2013},
и традиционно считаются одними из самых трудоёмких среди оптимизационных задач.
Их сложность экспоненциально растёт в зависимости от размерности пространства поиска \cite{Vavasis1995}.
Зачастую глобальныя оптимизация при размерности пространства в 10 переменных сложнее,
чем локальная оптимизация в существенно многомерном пространстве.
Для последней может оказаться достаточно применения простейшего метода градиентного спуска
или эвристических алгоритмов поиска по шаблону \cite{torczon1997},
в то время как чтобы \textit{гаранитрованно} отыскать глобальный оптимум методам
оптимизации приходится накапливать информацию о поведении целевой функции во всей области поиска
\cite{Jones2009,Paulavicius2011,Evtushenko2013,Strongin2000}.





\subsection*{Актуальность темы исследования}

\subsection*{Цель диссертационной работы}


Объект исследования –

Предмет исследования – м

\subsection*{Обоснование специальности}

\subsection*{Научная новизна}

\subsection*{Практическая ценность работы}

\subsection*{Обоснованность и достоверность результатов}

% Основные положения, выносимые на защиту

\subsection*{Реализация результатов работы}

\subsection*{Структура и объемы работы}

Квалификационная работа включает в себя введение, три главы, заключение и список литературы.
Полный объём квалификационной работы составляет 35 страниц, включая 11 рисунков и 6 таблиц.
Список литературы содержит 37 позиций.

\subsection*{Личный вклад автора}
