\section{Введение}

Я, Соврасов Владислав Валерьевич, аспирант третьего года обучения проходил исследовательскую
практику в период с 25.02.2020 по 06.07.2020 г. в ННГУ им. Н.И. Лобачевского, факультет
Институт информационных технологий математики и механики, кафедра математического обеспечения и супер компьютерных технологий.
Исследовательская практика выполняет системообразующую роль в образовательно-профессиональной подготовке специалиста высшей квалификации, позволяет выпускнику университета успешно выполнять основные функции педагога-исследователя в современном образовательном учреждении.
Исследовательская практика является одним из наиболее сложных и многоаспектных видов учебной работы аспирантов. Деятельность аспирантов в период практики является аналогом профессиональной деятельности педагога-исследователя, так как адекватна ее содержанию и структуре и организуется в условиях реального исследования.
Целью практики является подготовка аспирантов к осуществлению профессиональной исследовательской деятельности; систематизация, расширение и закрепление профессиональных знаний и умений; формирование исследовательской культуры.
Задачи практики:
\begin{itemize}
  \item Разработка и реализация параллельного алгоритма глобальной оптимизации
  для одновременного решения множества многоэкстремальных задач с
  невыпуклыми ограничениями.
  \item Исследование свойств сходимости нового алгоритма.
  \item Экспериментальная апробация разработанного алгоритма при
решении серии скалярных задач, возникающих при решении задач
многокритериальной многоэкстремальной оптимизации.
  \item Построение численных оценок эффективности алгоритма.
\end{itemize}

\section{Описание предметной области исследования}

Нелинейная глобальная оптимизация невыпуклых функций традиционно считается одной из самых трудных
задач математического программирования. Отыскание глобального минимума функции от нескольких переменных
зачастую оказывается сложнее, чем локальная оптимизация в тысячемерном пространстве. Для последней может оказаться достаточно
применения простейшего метода градиентного спуска, в то время как чтобы \textit{гарантированно} отыскать глобальный оптимум методам
оптимизации приходится накапливать информацию о поведении целевой функции во всей области поиска \cite{Jones2009,Paulavicius2011,Evtushenko2013,Strongin2000}. Решение серии таких задач при ограниченных вычислительных
ресурсах является еще более сложной задачей: помимо поиска глобального экстремума необходимо
распределять вычислительные ресурсы так, чтобы сразу во всех решаемых задачах положение глобального
экстремума было оценено примерно с одинаковым качеством. Обычно серию из \(q\) задач решают либо последовательно, либо
параллельно порциями по \(p,\:p<<q\) задач, где \(p\) --- количество параллельных вычислительных устройств.
Такой подход ведет к тому, что в каждый момент времени до окончания вычислений
остаются задачи, в которых оценка глобального оптимума не получена вообще, в то время, как в задачах из начала
списка оптимум может быть оценен даже с избыточной точностью.

В данной практической работе рассматривается обобщение ранее разработанного в ННГУ им. Н. И. Лобачевского
параллельного метода глобальной оптимизации для одновременного решения множества задач \cite{BarkalovStrongin2018} на
случай задач с нелинейными ограничениями. Для учета ограничений используется индексная схема \cite{Strongin2000},
позволяющая работать с частично вычислимыми целевым функциями и обладающая экономичностью,
сравнимой с другими подходами \cite{BarkalovLebedev2017}. Эффективность реализованного
алгоритма показана на примере решения множеств задач, сгенерированных специализированным
механизмом, порождающим наборы задач заданной размерности с заданным количеством нелинейных ограничений \cite{GergelBarkalov2019}.
Кроме искусственно сгенерированных задач, рассматриваемый метод протестирован также
на множестве задач, возникающем при решении задачи многокритериальной оптимизации
с нелинейными ограничениями методом свертки критериев \cite{Ehrgott2005}.
